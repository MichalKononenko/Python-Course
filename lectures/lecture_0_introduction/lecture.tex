\subsection{The First Shpeal}

Welcome to the first lecture on Python. This course is intended for people who
have no experience in programming to learn porgramming concepts, and get an
overview of what programming is. This course is not intended to be a 
comprehensive introduction to the Python language, or a nitty-gritty
discussion of the internals of any tools, although your homework assignments
will make use of Python. For this, there exists a multitude of books, covering
each subject in far greater detail than the time or expertise I have.

However, what I do want you to get out of this course are two truths that, for
the purposes of this course, I hold to be self-evident.

\textbf{Humans deal with complexity through abstraction.} Consider the problem
of image analysis. Were someone asked to recite, by memory, the light levels
in a $1000$ by $1000$ array of pixels in a photograph, it would be an
impressive feat to recite even the first line. Yet give them $1000$ discrete
$1000$ by $1000$ pixel arrays, and sorting them as to whether the picture
contains a cat or not proves trivial. This is because, at some point,
the human brain abstracts a particular collection of pixels into a line,
a set of lines are then abstracted into a shape. A set of shapes are then
abstracted into a cat. In this way, a complex problem becomes simpler to
reason about. The challenge of programming, therefore, is to produce good
abstractions that allow a programmer to express and solve their problem.

\textbf{Programming is a social activity.} Programmers don't work in a vacuum.
By engaging in programming, you join a community ranging from hobbyists to
professional developers. Programmers engage themselves with tasks ranging from
controlling hardware to transaction processing to networking. Many problems
that you will encounter will require solutions proposed by programmers that
have come before you. Outside of this community, you will be interacting with
people from all walks of life to put solutions together. Because of this, the
challenge of programming is not merely to coax a machine into producing a 
desired behaviour, but also in communicating that solution to others.

